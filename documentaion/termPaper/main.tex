\documentclass[conference]{IEEEtran}
\IEEEoverridecommandlockouts
% The preceding line is only needed to identify funding in the first footnote. If that is unneeded, please comment it out.
\usepackage{cite}
\usepackage{amsmath,amssymb,amsfonts}
\usepackage{algorithmic}
\usepackage{graphicx}
\usepackage{textcomp}
\usepackage{xcolor}
\def\BibTeX{{\rm B\kern-.05em{\sc i\kern-.025em b}\kern-.08em
    T\kern-.1667em\lower.7ex\hbox{E}\kern-.125emX}}

\begin{document}

\title{{Security issues in Dutch Auction}
\thanks{Term Paper submitted for CE/CZ4153 Blockchain Technology, NTU.}
}

\author{\IEEEauthorblockN{Chen Yiting}
\IEEEauthorblockA{U2022489K \\
\textit{SCSE}\\
\textit{Nanyang Technological University}\\
Singapore \\
CHEN1421@e.ntu.edu.sg}
\and
\IEEEauthorblockN{Zhao Yu}
\IEEEauthorblockA{U2023221D \\
\textit{SCSE}\\
\textit{Nanyang Technological University}\\
Singapore \\C200225@e.ntu.edu.sg}
\and
\IEEEauthorblockN{Zou Zeren}
\IEEEauthorblockA{U2022422H \\
\textit{SCSE}\\
\textit{Nanyang Technological University}\\
Singapore \\
ZZOU002@e.ntu.edu.sg}
}

\maketitle

\begin{abstract}
This study delves into key security issues in the solidity dutch auction system submitted for CE/CZ4153 Blockchain Technology course, with focused discussions on code quality, front-running attacks, and possible timestamp manipulation. Firstly, the paper discusses some vulnerabilities found in the code implementation by  utilizing Slither for a thorough examination of smart contracts. Secondly, the paper underscores the absence of privacy provisions in auction contracts, which may lead to the exposure of private data and bidder information. It also looks into potential weak spots with block timestamps in the auction implementation, highlighting how such manipulation could impact the fairness of auctions. The intent of this research is to improve the auction system developed by addressing these security concerns and suggesting solutions. The findings hold significance for creating more secure, efficient, and dependable dutch auction smart contracts.

%This is a template for the term paper of CE/CZ4153 Blockchain Technology course offered at the School of Computer Science and Engineering, Nanyang Technological University, Singapore. The paper should follow a similar style and format to incorporate all the necessary points of introduction, motivation, literature survey, observations, analysis, and solution to the issue. The paper should be 4 to 6 pages in length, including references. In case extra material is needed for analysis or arguments, the authors may include that as appendices, after the references.
\end{abstract}

\begin{IEEEkeywords}
blockchain, Ethereum, smart contracts, tokens, dutch auction, privacy.
\end{IEEEkeywords}

\section{Introduction}

\subsection{Blockchian Technologies}
Blockchain technologies, essential for secure distributed computations, are gaining traction due to their wide range of applications\cite{javaidBlockchainTechnologyApplications2021}. Ethereum, a popular blockchain platform, supports not just cryptocurrency but also other applications like games and financial services through its decentralized Ethereum Virtual Machine (EVM) and smart contracts\cite{pothavarjulaInvestigationDecentralizedLedger2022}. Three fundamental concerns are critical in the field of blockchain: security, scalability, and privacy. 

\begin{itemize}
    \item \textbf{Security} properties of blockchain stems from developments in both cryptography and chain architecture. Several intrinsic security features, including consistency, resistance to tampering, resistance to Distributed Denial-of-Service (DDoS) attacks, pseudonymity, and resistance to double-spending attacks, are ensured by the design of the blockchain\cite{zhangSecurityPrivacyBlockchain2020}. However, extra high-level security features are needed with the advancement of smart contracts, such as overflow protection and code injection resistance.

    \item \textbf{Privacy} concerns, particularly in applications that manage sensitive data, stem from blockchain's transparent nature. To prevent user data from being misused or exposed, it can be difficult to strike a balance between transparency and confidentiality.
    
    \item \textbf{Scalability} refers to the ability of a blockchain network to effectively manage high transaction volumes. With poor throughput, significant transaction delay, and massive consumption of energy, both Bitcoin and Ethereum are experiencing scalability problems\cite{zhouSolutionsScalabilityBlockchain2020}.
    
\end{itemize}

\subsection{Dutch Auction}
In an auction, a seller offers products or services for sale, and potential buyers submit their bids based on the price they are willing to pay. Numerous types of auctions have been developed over time. In a Dutch auction, the seller sets the starting price and gradually reduces it until a bidder agrees to the going rate\cite{bennettGoingGoingGone2020}.

\subsection{Project Focus and Development}

Our development project's main goal was to create an dutch auction system based on blockchain technology, taking advantage of the immutability and decentralisation of blockchain technology to guarantee an open and equitable bidding process. In order to support different kinds of auctions and preserve the integrity and dependability of the bidding process, we created a number of smart contracts.
 

\subsection{Paper Focus and Contribution}

This paper focuses on blockchain \textbf{security} in our auction system, discussing front-running attack issues in smart contracts, potential timestamp manipulation, and overall code quality and safety. Despite the importance of privacy and scalability, enhancing the security framework is essential. The study addresses these issues, provides solutions, and contributes to the stability and reliability of our dutch auction system, setting roads to improvement for future blockchain applications.

%In this section, you should write a few paragraphs on blockchain (brief, can adapt from lectures, in your own words), the issues of security, privacy and scalability in blockchain (brief, can adapt from lectures, in your own words), what your development project topic was, and what you finally developed.

%You should also mention exactly which issue out of the three Security, Privacy, Scalability you are going to present, and what would be the overall contribution of this paper. Argument as why this issue is the most important is not required here.

\section{Motivation and Literature Survey}

\subsection{Motivation}

Security is the paramount concern for our blockchain-based, decentralized auction system. This emphasis is due to the financial implications of the system, the immutable nature of smart contracts, and lessons learned from compromised blockchain projects.

Firstly, the system's financial transactions, managed by smart contracts, require high security standards to prevent loss or unfair advantages. Secondly, in a decentralized environment, trust lies in the code, making its security vital for user confidence and system integrity\cite{zachariadisGovernanceControlDistributed2019}. Blockchain's immutability and transparency further magnify security concerns. Any flaws become permanent vulnerabilities if not addressed beforehand, necessitating rigorous security measures during development. Lastly, past blockchain projects reveal that security is often overlooked, leading to breaches and system failures\cite{hwangGapTheoryPractice2020}.

Therefore, despite the importance of privacy and scalability, security is our top priority. This aligns with the needs of our users, stakeholders, and best practices within the blockchain community.

%In this subsection, you should clearly argue which one of the three issues -- Security, Privacy, Scalability -- concerns you the most in case of the Decentralized Application you developed. You may refer to the lectures, invited talks, related works, or any other instance of similar development projects to argue this.

\subsection{Literature Survey}

Various blockchain solutions have been proposed in recent studies to improve the effectiveness and security of e-auction systems. These studies investigate various facets of blockchain technology, ranging from enhancing the auction procedure to guaranteeing transaction security.

\textbf{Blockchain Auction for Secure Communications:}
Khan et al. \cite{khanBlockchainBasedDistributiveAuction2019} propose a blockchain-based distributive auction system designed for relay-assisted secure communications. Their system eliminates the need for a central authority by decentralising the auction process through the use of blockchain. By automating transaction rules, smart contract implementation improves the security and stability of the system. Additionally, the study discusses possible weaknesses in distributed systems and suggests defences against different kinds of malicious behaviours.

\textbf{Blockchian Auction in UAE:}A blockchain-based e-auction system designed specifically for the United Arab Emirates (UAE) market is presented by Qusa et al\cite{qusaSecureEAuctionSystem2020}. Their research focuses on leveraging blockchain technology to establish an open and impenetrable bidding environment. The technology uses smart contracts to automate auction procedures, eliminating the need for middlemen. The authors talk about how this system can deal with issues that frequently arise in online auctions, like bid rigging and privacy concerns.

\textbf{Enhanced Tree-Structured E-Auction:}
An enhanced blockchain tree structure-based e-auction system is proposed by Sarfaraz et al. \cite{sarfarazTreeStructurebasedImproved2021} to make the processes of evaluating bids and choosing winners more efficient. By reducing the complexity of bid management, the tree-structured approach facilitates the safe and transparent handling of a large number of bids. The study demonstrates how blockchain technology can simplify e-auction procedures while maintaining security and fairness.

\textbf{Safe Sealed-bid Bid Procedure:}
Zhang and colleagues \cite{zhangSSBASFASecureSealedbid2022} created a blockchain-based secure sealed-bid auction system. Their method makes use of blockchain's transparency and immutability to guarantee the validity of auction bids. By enforcing stringent guidelines for bid submission and opening, the system uses smart contracts to prevent bid leakage and manipulation. The authors do point out the difficulties in ensuring that bidders correctly interact with the contract functions and the complexity of implementing smart contracts in such a system.

%In this subsection, you should carefully curate similar works in the area of your interest, with proper citation (see references). This may include similar development projects or decentralized applications that have faced the same issues, lectures, articles, books or papers that talk about the issue in your case, or any other academic material related to your specific case.


\section{Observations and Analysis}
This section delves into a comprehensive risk analysis of the Dutch Auction smart contract. The aim is to identify potential vulnerabilities and threats that could compromise the system's integrity, fairness, and functionality. By systematically examining each component of the auction system, we can not only uncover inherent weaknesses but also propose effective countermeasures to safeguard against these risks in the Proposed Solutions\ref{sec:solutions} section.

\subsection{Slither Analysis of Code Quality and Safety Issues}
Beginning with a thorough examination using Slither to evaluate the code quality and identify potential issues, we discovered two primary concerns within the auction's smart contract, strict equalities and low-level calls.
\subsubsection{Dangerous Strict Equalities}
The smart contract's \texttt{DutchAuction.\_preValidateFinalization()} function uses strict equality checks (i.e., \texttt{==}) to compare \texttt{remainingSupply()} with 0. This approach is prone to logic misinterpretation due to potential rounding or calculation errors inherent in blockchain computations. Failure in this check could lead to the auction not finalizing correctly, resulting in funds being locked in the contract, similar to the ‘Gridlock’ vulnerability found in Edgeware's smart contracts \cite[Sec. 2.3, p. 15]{trailofbits2019gridlock}.


\subsubsection{Low-Level Calls}
The Auction contract employs low-level calls such as \texttt{.call()}, \texttt{.delegatecall()}, and \texttt{.staticcall()} for interactions with external contracts. These methods do not automatically revert on failure, which could lead to security risks like reentrancy attacks. The unchecked return values from these calls might result in fund loss or invalid contract states \cite[Chap. 5, p. 37]{zipfel2023lowlevelcall}.


\subsection{Front running attacks}
In a Dutch auction, the item's price typically starts high and decreases over time until a bid is accepted. This structure can make the auction particularly vulnerable to front running. Firstly, bids are visible on the network before they are confirmed. This visibility allows potential attackers to observe incoming bids. An attacker, upon seeing a bid, could quickly place a higher bid with a higher gas fee. The higher gas fee incentivizes miners to prioritize the attacker’s transaction over the original bid. If the attacker's bid is processed first due to the higher gas fee, it could be accepted by the smart contract before the original bidder’s transaction. This could lead to the attacker successfully acquiring the auctioned item at a price that might be lower than what the original bidder was willing to pay. Thus, the fairness of the auction would be undermined, as well as the integrity of the smart contract, potentially leading to a loss of trust among participants.

\subsection{Timestamp Dependency}
Generally, the execution of the smart contract depends on the timestamp of the current block, and if the timestamp is different, the execution result of the contract is also different. Obtaining a timestamp in a smart contract can only be done by relying on a node (miner). This means that the timestamp obtained in the contract is determined by the local time of the computer of the node (miner) running its code. Thus, this local time can be controlled by the miner. 
In the Dutch auction, where the price of an item decreases over time, a strategically manipulated timestamp could distort the competitive environment \cite[Sec. C.C, p. D]{clark2012security}. The timestamp of a block, accessed by \texttt{now} or \texttt{block.timestamp}, which is used for critical functionalities like determining the starting or ending time of the dutch auction, can be manipulated by a miner when they create a block. Miners possess the discretion to adjust the \texttt{block.timestamp} slightly, provided their adjustment remains within the constraints imposed by the Ethereum protocol \cite[Sec. A.A, p. B]{buterin2014ethereum}. If a miner publishes a block with a manipulated timestamp, it can unfairly benefit certain bidders. For instance, an earlier timestamp might close the auction prematurely, or a later one might give some bidders an unfair advantage if they can bid at a lower price.


\section{Proposed Solutions}\label{sec:solutions}
After a meticulous analysis of the identified vulnerabilities within the Dutch Auction smart contract, we now turn our attention to the development of effective solutions. This section is dedicated to presenting carefully crafted strategies aimed at addressing the specific challenges uncovered. Our solutions are not only designed to rectify the existing issues but also to enhance the overall robustness and security of the smart contract, thereby ensuring a fair and reliable auction process.

\subsection{Solution to Slither analysed issues}
\subsubsection{Solution for Strict Equalities vulnerability}
A more flexible approach, such as using range-based checks or near-zero conditions, is recommended. These methods account for minor calculation
discrepancies. Incorporating defensive programming practices
and using tools like Slither’s dangerous-strict-equality detector are essential to preemptively identify and mitigate such
vulnerabilities (see [12], Sec. 2.3, p. 172)

\subsubsection{Solution for Low-Level Calls vulnerability}
Mitigating risks associated with low-level calls in smart contracts involves using Solidity's \texttt{extcodesize} for contract existence verification. Robust exception handling and return value verification are essential for securing these interactions (refer to \cite{wohrer2018security}, Chap. 5, p. 89).


\subsection{Solution to Front-running attack}
There are several strategies to prevent front-running attack, however no single measurement can fully avoid the attack, in this section we would introduce three approaches which would best applicable in our dutch auction application.


Setting uniform gas prices for all transactions can create a more equitable environment by preventing bidders from outbidding each other through increased gas fees \cite[Sec. 3.6, p. 50]{reference2023}. Additionally, incorporating time locks or introducing delays in bid processing effectively narrows the window for potential front-running attacks \cite[Sec. 3.3, p. 47]{reference2023}. Moreover, employing on-chain randomness enhances bid processing order unpredictability, offering an additional layer of protection against front-running \cite[Sec. 3.8, p. 52]{reference2023}. 


By integrating these approaches, the likelihood of front-running in Dutch auctions can be significantly reduced, thereby upholding both fairness and security.

\subsection{Solution to Timestamp vulnerability}
Mitigating the timestamp vulnerability involves three strategies, the analysis of each is shown below. 

\subsubsection{Block Number Utilization}

Developers may opt for block numbers as a less manipulable alternative, leveraging the inherent sequential nature of block mining \cite[Sec. 2.3, p. 15]{clark2012security}. 
In the Ethereum blockchain, each block is sequentially numbered, and this count is known as the block number. The use of block numbers as a timekeeping mechanism is based on the predictable average time it takes to mine a block on the Ethereum network, which is roughly constant (approximately every 13–15 seconds for Ethereum). Therefore, developers can estimate the passage of time by counting blocks. For example, if they need a timer for one hour, they could set it to roughly 240 blocks (assuming 15 seconds per block) \cite[Sec. 6.1, p. 22]{wood2014ethereum}.

Block numbers provide a reliable measure of time due to their inherent sequential progression. Unlike timestamps, which can be influenced by miners within certain limits, block numbers increase strictly in order and cannot be altered by miners \cite[Sec. 4.2, p. 11]{wood2014ethereum}.

\subsubsection{Guard Clauses}

Guard clauses can automatically detect the anomalies which deviates significantly from the expected timeby comparing the timestamp of each new block against expected thresholds.
The primary purpose of these clauses is to deter and mitigate risks associated with timestamp manipulation. Malicious miners might attempt to adjust block timestamps to gain an unfair advantage, particularly in applications like smart contract-based auctions or financial transactions \cite[Sec. 4.5, p. 78]{wood2014ethereum}.

Implementing guard clauses requires careful consideration of the blockchain's operational characteristics, including block time variability and network latency. A well-designed guard clause will invalidate a block with a timestamp that falls outside a reasonable range, thus preventing its acceptance into the blockchain.

\subsubsection{External Oracles}

The use of external oracles can provide a more robust timing source \cite{chainlink2020}. 
Services such as Oraclize are designed to provide fair and tamper-proof random numbers, enhancing the security and fairness of smart contracts \cite[Sec. X.X, p. Y]{clark2012security}. Additionally, these services offer the benefit of verifiable randomness, which adds a layer of transparency to the process \cite[Sec. A.A, p. B]{bonneau2015research}.

On the downside, relying on such services often involves additional fees, increasing the operational cost \cite[Chap. C.C, p. D]{narayanan2016bitcoin}. Moreover, this reliance introduces new points of failure, potentially reducing the system's reliability \cite[Sec. E.E, p. F]{eyal2014majority}. The balance between these factors should be determined based on specific circumstances and requirements. 


\section{Conclusion}
In conclusion, this paper has extensively analyzed the vulnerabilities in Dutch auction system developed for the CE/CZ4153 Blockchain Technology
course at Nanyang Technological University, using Slither as a key tool for identifying crucial security concerns. The vulnerabilities highlighted include issues with strict equalities, the implementation of low-level calls, exposure to front-running attacks, and dependencies on timestamps. These vulnerabilities, if not appropriately addressed, pose significant risks to the functionality, fairness, and overall security of the auction system.

The findings of this analysis emphasize the critical importance of implementing defensive programming techniques and utilizing analytical tools like Slither for early detection of security flaws in smart contracts. Effective strategies to mitigate these risks include the adoption of range-based checks instead of strict equalities, pre-validation of contract existence prior to executing low-level calls, and the integration of thorough exception handling mechanisms. These measures are vital for enhancing the robustness and reliability of Dutch Auction smart contracts.




\bibliographystyle{ieeetr} 
\bibliography{ref}

\end{document}